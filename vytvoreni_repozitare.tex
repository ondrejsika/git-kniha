\section{Vytvoreni repozitare}

Repozitar se vytvari bud prikazem \lstinline|git init| a nebo se naklonuje. O klonovaci bude rec v kapotole vzdalene repozitare.

Pokud chceme v nekterem adresari vytvorit repozitar, staci zavolat prikaz \lstinline|git init|.

Priklad:

\begin{lstlisting}
sika@x1:~$ mkdir r
sika@x1:~$ cd r
sika@x1:~/r$ git init
Initialized empty Git repository in /home/sika/r/.git/
sika@x1:~/r (master)$
\end{lstlisting}

\subsection{Git status}

\lstinline|git status| je prikaz, ktery vypise stav vaseho pracovniho adresare a repozitare. Je velmi uzitecny. Po vytvoreni noveho repozitare vypise toto:

\begin{lstlisting}
sika@x1:~/r (master)$ git status
On branch master

Initial commit

nothing to commit (create/copy files and use "git add" to track)
\end{lstlisting}

