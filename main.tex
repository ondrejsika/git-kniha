\documentclass[12pt,a5paper]{article}
\usepackage[utf8]{inputenc}
\usepackage[margin=2cm]{geometry}
\usepackage[parfill]{parskip}
\usepackage{listings}
\lstset{
   breaklines=true,
   basicstyle=\ttfamily}

\title{Git kniha}
\author{Ondrej Sika}
\date{2015}

\begin{document}
\maketitle

\newpage

$$$$

\vfill

{\LARGE Git kniha}
\vspace{0.3cm}

{\large Ondrej Sika}\\
\texttt{ondrej@ondrejsika.com}\\
\texttt{http://ondrejsika.com}
\vspace{0.8cm}

Domovska stranka knihy je\\
\texttt{https://ondrejsika.com/books/git-kniha}

\newpage

$$$$

\tableofcontents



\section{Predmluva}
\section{Co je to GIT}
\section{Vytvoreni repozitare}
\section{3 stavy}
\section{Commit}
\section{Vetve}
\section{Merge}
\section{Rebase}
\section{Vzdalene repozitare}
\section{Github}

%\begin{thebibliography}{10}
%  \bibitem{python-kniha}Ondrej Sika:
%    {\em Python kniha}.\\
%    Sika Press, Praha, 2015.\\
%    ISBN 80-85867-35-4.\\
%    \lstinline|https://ondrejsika.com/books/python-kniha|
%  \bibitem{homesim}Ondrej Sika:
%    {\em Latex paper examples}. (brezen 2015).\\
%    \lstinline|https://ondrejsika.com/others/latex/|
%  \bibitem{wiki-shell} Wikipedia:
%    \emph{Shell (Programovani)}. (brezen 2015).\\
%    \lstinline|https://cs.wikipedia.org/wiki/Shell_(programov%C3%A1n%C3%AD)|
%\end{thebibliography}
\end{document}

