\documentclass[12pt,a5paper]{article}
\usepackage[utf8]{inputenc}
\usepackage[margin=2cm]{geometry}
\usepackage[parfill]{parskip}
\usepackage{listings}
\lstset{
   breaklines=true,
   basicstyle=\ttfamily}

\title{Git kniha}
\author{Ondrej Sika}
\date{2015}

\begin{document}
\maketitle

\newpage

$$$$

\vfill

{\LARGE Git kniha}
\vspace{0.3cm}

{\large Ondrej Sika}\\
\texttt{ondrej@ondrejsika.com}\\
\texttt{http://ondrejsika.com}
\vspace{0.8cm}

Domovska stranka knihy je\\
\texttt{https://ondrejsika.com/books/git-kniha}

\newpage

$$$$

\tableofcontents



\section{Predmluva}

Proc jsem se rozhodl psat knizku o Gitu, kdyz je jich vsude plno? A kdyz je i slavna Pro Git od Scotta Sharona prelozena a zdarma dostupna na \lstinline|https://knihy.nic.cz/|. Tyto knizku jsou urcene programatorum a vetsina jich je hodne pokrocilich, ja se v teto knize budu snazit ukazat jak Git funguje prevazne neprogramatorum pripadne zacinajicim programatorum. Verim ze Git je vhodny i pro lidi, kteri napriklad pisi nejake prace, prezentace a nemusi byt vubec programatori.

Git ma mnoho grafickych nastaveb, ale ja verim, ze pokud si clovek osaha ovladani z terminalu, nakonec to pro nej bude nejlepsi. Proto v teto knizce budu vse ovladat z terminalu, shlellu a nebudu se zabyvat klikatky. Pokud je pro vas Shell neco uplne noveho, kouknete se na {\bf Ondrej~Sika:~Shell~kniha} (\lstinline|https://ondrejsika.com/books/shell-kniha|). Budu se ovsem snazit nejnutnejsi minimum popsat i v teto knize.

\section{Co je to Git}

Git je verzovaci system, ktery je primarne urcen pro verzovani kodu. Ale muzeme jim verzovat prakticky cokoliv. Git je velmi rozsireny (dnes uz majoritni) verzovaci system a pro kazdeho programatora je jeho znalost naprostou nutnosti. Jeho hlavni vyhody jsou moznost spoluprace nad jednim kodem ve vice lidech, prace na ruzmych castech kodu bezpecne v oddelenych vetvich a revize kodu.

\section{Vytvoreni repozitare}

Repozitar se vytvari bud prikazem \lstinline|git init| a nebo se naklonuje. O klonovaci bude rec v kapotole vzdalene repozitare.

Pokud chceme v nekterem adresari vytvorit repozitar, staci zavolat prikaz \lstinline|git init|.

Priklad:

\begin{lstlisting}
sika@sika-mol:~$ mkdir myrepo
sika@sika-mol:~$ cd myrepo
sika@sika-mol:~/myrepo$ git init
Initialized empty Git repository in /home/sika/myrepo/.git/
sika@sika-mol:~/myrepo (master)$ 
\end{lstlisting}

\section{Gitignore}

Pred tim, nez zacneme s repozitarem pracovat, je dobre Gitu rici o jake soubory se nema zajimat. Jsou to soubory docasne, napriklad zalohy editoru, nastavenu prostredi ci zkompilovane programy. Ty verzovat nemusime a ani nechceme. Proto je vhodne zjistit, jake soubory v repozitari nechceme jeste pred tim, nez s nim zacneme pracovat. Napriklad, kdyz budu psat knizku v LaTeXu tak chci verzovat jen .tex soubory a vystupni .pdf ne (ten si kazdy vytvori sam z .tex zdroju). LaTeX ale pri generovani vytvari jeste .aux, .log a .toc. Dale vim, ze budu knizku psat ve Vimu, ktery vytvari soubory .swp. Vsechny proto napisu do souboru \lstinline|.gitignore| (tecka na zacatku znamena v Unix svete skryty soubor/slozka). Kdyz chci ignorovat vsechny soubory s priponou .pdf, napisi do gitignore \lstinline|*.pdf| Nas soubor bude potom vypadat takto: 

\begin{lstlisting}
*.swp
*.aux
*.log
*.toc
*.pdf

\end{lstlisting}


\section{3 stavy}
\section{Commit}
\section{Vetve}
\section{Merge}
\section{Rebase}
\section{Vzdalene repozitare}
\section{Github}

%\begin{thebibliography}{10}
%  \bibitem{python-kniha}Ondrej Sika:
%    {\em Python kniha}.\\
%    Sika Press, Praha, 2015.\\
%    ISBN 80-85867-35-4.\\
%    \lstinline|https://ondrejsika.com/books/python-kniha|
%  \bibitem{homesim}Ondrej Sika:
%    {\em Latex paper examples}. (brezen 2015).\\
%    \lstinline|https://ondrejsika.com/others/latex/|
%  \bibitem{wiki-shell} Wikipedia:
%    \emph{Shell (Programovani)}. (brezen 2015).\\
%    \lstinline|https://cs.wikipedia.org/wiki/Shell_(programov%C3%A1n%C3%AD)|
%\end{thebibliography}
\end{document}

