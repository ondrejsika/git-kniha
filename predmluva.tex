\section{Predmluva}

Proc jsem se rozhodl psat knizku o Gitu, kdyz je jich vsude plno? A kdyz je i slavna Pro Git od Scotta Sharona prelozena a zdarma dostupna na \lstinline|https://knihy.nic.cz/|. Tyto knizku jsou urcene programatorum a vetsina jich je hodne pokrocilich, ja se v teto knize budu snazit ukazat jak Git funguje prevazne neprogramatorum pripadne zacinajicim programatorum. Verim ze Git je vhodny i pro lidi, kteri napriklad pisi nejake prace, prezentace a nemusi byt vubec programatori.

Git ma mnoho grafickych nastaveb, ale ja verim, ze pokud si clovek osaha ovladani z terminalu, nakonec to pro nej bude nejlepsi. Proto v teto knizce budu vse ovladat z terminalu, shlellu a nebudu se zabyvat klikatky. Pokud je pro vas Shell neco uplne noveho, kouknete se na {\bf Ondrej~Sika:~Shell~kniha} (\lstinline|https://ondrejsika.com/books/shell-kniha|). Budu se ovsem snazit nejnutnejsi minimum popsat i v teto knize.

