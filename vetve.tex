\section{Vetve}

Nejdrive co jsou to vetve. Muzeme si to predstavit ze pracujeme na 2 ruznych castich projektu aniz by se mezi sebou ovlivnovali. Jsou nezavisle az do te doby co se je rozhodneme sloucit. Prace s vetvemi je dulezita a git ji ma vyresenou velmi efektivne a je rychla.

Doted jsme pracovali pouze ve vetvi {\bf master}. Prikaz \lstinline|git branch| nam vypise lokalni vetve a oznaci nam hvezdickou aktualni vetev.

\begin{lstlisting}
sika@x1:~/r (master)$ git branch
* master
\end{lstlisting}

Jak si muzete vsimnout, muj prompt shellu sam ukazuje aktualni vetev, v zavorkach.

Pokud chci vytvorit novou vetev, pouziji prikaz \lstinline|git branch <name>| kde name je nazev nove vetve. Vetve je rozumne pojmenovavat podle toho co v nic chceme delat, ale nazev se vzdy da zmenit.

Pri praci s vetvemi doporucuji mit cisty working directory, toho muzeme jednoduse docilit prikazem \lstinline|git stash|, jak uz vime.

\begin{lstlisting}
sika@x1:~/r (master)$ git branch experimental
sika@x1:~/r (master)$ git branch
  experimental
* master
sika@x1:~/r (master)$
\end{lstlisting}

Vetev jsme sice vytvorili, ale porad zustavame na te puvodni. V nekdterych pripadech to plati, ze si vytvorime vetev jen jako zalohu a pokracujeme v praci, ale v typickem pripade je to obracene. Vytvarime novou vetev abychom v ni delali nejakou slozitejsi praci.

Pro prepnuti pouzijeme prikaz \lstinline|git checkout <name>|, ktery nas prepne na danou vetev.

\begin{lstlisting}
sika@x1:~/r (master)$ git checkout experimental
Switched to branch 'experimental'
sika@x1:~/r (experimental)$ git branch
* experimental
  master
sika@x1:~/r (experimental)$
\end{lstlisting}

Pokud se chceme prepnot do neexistujici vetve, git vypise chybu:

\begin{lstlisting}
sika@x1:~/r (master)$ git checkout new
error: pathspec 'new' did not match any file(s) known to git.
\end{lstlisting}

Kdyz ovsem chceme vetev vytvorit a rovnou se do ni prepnout, coz je podle me nejcastejsi pouziti, pouzijeme prikaz git checkout s parametrem -b:

\begin{lstlisting}
sika@x1:~/r (master)$ git branch
* master
sika@x1:~/r (master)$ git checkout -b new
Switched to a new branch 'new'
sika@x1:~/r (new)$ git branch
  master
* new
sika@x1:~/r (new)$
\end{lstlisting}

Uz vime jak vetve vytvaret a prepinat, jeste je potreba umet vetev smazat. Prikaz pro smazani vetve je \lstinline|git branch -d <name>|. Nemuzeme ovsem smazat vetev na ktere se aktualne nachazime.

\begin{lstlisting}
sika@x1:~/r (new)$ git branch
  master
* new
sika@x1:~/r (new)$ git checkout master
Switched to branch 'master'
sika@x1:~/r (master)$ git branch -d new
Deleted branch new (was 3a612f1).
sika@x1:~/r (master)$ git branch
* master
\end{lstlisting}

Ted uz je na case zacit vetve pouzivat, uvidite, ulehci vam praci. Umozni vam totiz v jedne vetvi prestat a v jine pokracovat na uplne jine casti vaseho projektu. Klidne muzete ve dvou vetvich delat to a same najednou a potom vybrat tu vetev s lepsim resenim a pouzit ji a druhou smazat.


\subsection{Merge}

Merge je jednim ze zpusobu jak sloucit 2 vetve. Merge vytvori novy commit se 2 rodici a slouci tak 2 vetve. Pokud se ve vetvich neupravovala konfliktni cast projektu (stejna cast jednoho souboru) pak probehne merge bez komplikaci.

\begin{lstlisting}
% TODO merge
\end{lstlisting}


\subsection{Rebase}

